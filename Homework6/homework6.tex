\documentclass{article}
\usepackage{../csci-246-fall2018/hw/template/fasy-hw}
\usepackage{amsmath}
\usepackage{cancel}
\usepackage{hyperref}

\author{Nathan Stouffer}
\problem{1-2}
% \problem{A-B} means Problem Set A, Problem B.
\collab{none}
% or give names, e.g., \collab{Alyssa P. Hacker and A. Student}

\begin{document}
Prove $A(n)=\sum\limits_{i=0}^n F_i = F_{n+2}-1$. \\\\
Background: $F_0=F_1=1$ and $F_n=F_{n-1}+F_{n-2}$ $\forall n \geq 2$. \\\\
\begin{proof}
	We will prove this by mathematical induction. \\\\
	Base case for $n=2$: \\
	$\sum\limits_{i=0}^n F_i =\sum\limits_{i=0}^2 F_i = F_0+F_1+F_2=1+1+2=4$ \\\\
	$F_{n+2}-1=F_{2+2}-1=F_4-1=(F_3+F_2)-1=((F_2+F_1)+(F_0+F_1))-1=(F_1+2*(F_0+F_1))-1=(1+2*(1+1))-1=(1+2*2)-1=4$ \\\\
	We will now assume $A(n)$ is true and use this to prove $A(n+1)$: \\
	\begin{align*}
		\sum\limits_{i=0}^{n+1}F_i&=F_{n+1}+\sum\limits_{i=0}^{n}F_i \\
		&=F_{n+1}+(F_{n+2}-1) &\text{by inductive assumption} \\
		&=(F_{n+1}+F_{n+2})-1 &\text{by associativity} \\
		&=F_{n+3}-1 &\text{by definition of Fibonacci Sequence} \\
		A(n+1)&=F_{(n+1)+2}-1
	\end{align*}
	Therefore, by inductive assumption, $A(n)$ is proved to be true.
\end{proof}

\problem{1-3}
\collab{none}
\clearpage
\header
\section*{Section 5.5, Problem 42}
Prove $A(n)=\prod\limits_{i=1}^n(c*a_i)=c^n\prod\limits_{i=1}^na_i$.
\begin{proof}
	We will prove this by mathematical induction. \\\\
	Base case for $n=2$: \\
	$\prod\limits_{i=1}^n(c*a_i)=\prod\limits_{i=1}^2(c*a_i)=(c*a_1)*(c*a_2)=c^2*a_1*a_2=c^2\prod\limits_{i=1}^2a_i$ \\\\
	We will now assume $A(n)$ is true and use this to prove $A(n+1)$: \\
	\begin{align*}
		A(n+1)&=\prod\limits_{i=1}^{n+1}(c*a_i) \\
		&=(c*a_{n+1})\prod\limits_{i=1}^n(c*a_i) \\
		&=(c*a_{n+1})(c^n\prod\limits_{i=1}^na_i) &\text{by inductive assumption} \\
		&=c^{n}*c*a_{n+1}\prod\limits_{i=1}^{n+1}a_i &\text{by associativity} \\
		A(n+1)&=c^{n+1}\prod\limits_{i=1}^{n+1}a_i
	\end{align*}
	Therefore, by inductive assumption, $A(n)$ is proved to be true.
\end{proof}

\problem{1-4}
\collab{none}
\clearpage
\header
\section*{Section 5.5, Problem 7}
$u_k=k*u_{k-1}-u_{k-2}$ $\forall k \geq 3$ \\\\
$u_1=1$ \\
$u_2=1$ \\
$u_3=3*u_{3-1}-u_{3-2}=3*u_2-u_1=3*1-1=2$ \\
$u_4=4*u_{4-1}-u_{4-2}=4*u_3-u_2=4*2-1=7$

\section*{Section 5.5, Problem 32}

\problem{1-5}
\collab{none}
\clearpage
\header
\begin{enumerate}[5.1]
	\item Euler's Formula: $e^{i\theta}=cos(\theta)+i*sin(\theta)$
	\item Cube: $F_0=6$, $E_0=12$, $V_0=8$ \\
	Cutting a tetrahedron off each corner: \\
	The amount of faces increases by the number of vertexes: $F_1=F_0+V_0=6+8=14$ \\
	The amount of edges increases by 3 times the number of vertexes: $E_1=E_0+3V_0=12+3*8=12+24=36$ \\
	The amount of vertexes is 3 times the number of vertexes: $V_1=3V_0=3*8=24$
\end{enumerate}

\problem{1-6}
\collab{none}
\clearpage
\header

\section*{Leonhard Euler}

References to online resources are provided as footnotes. \\

Leonhard Euler was a Swiss mathematician in the 1700s who is famous for discovering many theorems in number theory such as $e^{i\theta} = cos(\theta)+i*sin(\theta)$ and $F-E+V=2$. According to the Encyclopedia Britannica, Euler "threw new light on nearly all parts of pure mathematics. "
\footnote{\url{https://www.britannica.com/biography/Leonard-Euler}}
Euler's ideas continue to contribute to modern theorems in mathematics and computing.

\end{document}

